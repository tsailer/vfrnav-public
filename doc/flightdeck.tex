%% pdflatex or latex
\newif\ifphymodel
\phymodeltrue

\newif\iftexht
\newif\ifmypdf
\ifx\HCode\UnDef
  \texhtfalse
  \ifx\pdfoutput\undefined
    \mypdffalse
    \texhtfalse
  \else
    \pdfoutput=1
    \mypdftrue
  \fi
\else
  \mypdffalse
  \texhttrue
\fi

\newif\ifregssubsection
\regssubsectiontrue

\ifmypdf
\documentclass[a4paper,10pt,pdftex]{article}
\else
\documentclass[a4paper,10pt,dvips]{article}
\fi
\usepackage[german,english]{babel}
\selectlanguage{english}
\usepackage{graphicx}
\usepackage[isolatin]{inputenc}
\usepackage{array}
\usepackage{amsmath}
\usepackage{amssymb}
\frenchspacing
%\pagestyle{plain}
\pagestyle{headings}
%\usepackage[a4paper,ignoremp]{geometry}
\usepackage[a4paper,ignoremp,left=2cm,right=2cm,top=3cm,bottom=3cm]{geometry}
\usepackage{rotating}
\usepackage{hhline}
\usepackage{multirow}
\usepackage{pdflscape}
\usepackage{longtable}
\usepackage{afterpage}
%
\clubpenalty=4000
\widowpenalty=4000
%

\ifmypdf
\usepackage[pdftex]{color}

% The hyperref package makes sure that all references in your
% document become colored clickable links.
%\usepackage[colorlinks,hyperindex]{hyperref}
\usepackage[hyperindex]{hyperref}
  \hypersetup{pdftitle={FlightDeck Manual},
    pdfsubject={Flightdeck},
    pdfauthor={Thomas M. Sailer},
    pdfkeywords={Flightdeck},
    bookmarksdepth=paragraph,
%% pdfpagemode={FullScreen}
}

% supress the ugly colored boxes around the active parts of the
% text.
\def\pdfBorderAttrs{/Border [0 0 0] } % No border around Links
% Some further Color tuning. Use xcolorsel for help with the colors
\definecolor{links}{rgb}{0.2116,0.0104,0.7716} % BlueViolet
\def\LinkColor{links}
\definecolor{anchors}{rgb}{0.5812,0.0665,0.0659} % IndianRed
\def\AnchorColor{anchors}

\hypersetup{bookmarks=true,bookmarksopen=true,bookmarksopenlevel=1,bookmarksnumbered=true}

%\DeclareGraphicsRule{.tiff}{tif}{.tiff}{}
%\DeclareGraphicsExtensions{.jpg,.pdf,.tif}

\else
%\usepackage[hyperindex,dvips]{hyperref}
\usepackage[hyperindex,hypertex]{hyperref}
\def\pdfBorderAttrs{/Border [0 0 0] } % No border around Links
\fi
\usepackage{colortbl}

%
\makeatletter
\newcommand{\jnxbookmark}[1]{\ifmypdf
\ifx\@currentHref\@empty\phantomsection\fi
\Hy@writebookmark{}{#1}{\@currentHref}{-1}{toc}\fi}
\makeatother
%
%\input{psfig}
%
\newcommand{\jnxfig}[3][]{\ifmypdf\includegraphics[#1]{#3}
  \else\includegraphics[#1]{#2}\fi}
%

\begin{document}
\sloppy

\title{FlightDeck}
\author{Thomas M. Sailer}

%
\makeatletter
\newpage\thispagestyle{empty}
\noindent\begin{minipage}{\textwidth}
  \flushright
  {\sffamily\bfseries\Huge\@title}
  \noindent\rule[-1ex]{\textwidth}{5pt}\\[5ex]
  {\sffamily\Large\@author}\\[1.5ex]
  {\sffamily\Large\@date}\\[5ex]
  {\sffamily\Large draft}\\[5ex]
\end{minipage}
\makeatother
%
\tableofcontents

\newcommand\graycol{\iftexht\else\cellcolor[gray]{.9}\fi}


\section{Introduction}

This manual describes the VFRNav package. VFRNav consists of the
following main programs:

\begin{itemize}
\item VFRNav
\item FlightDeck
\end{itemize}

VFRNav is optimized for small resolution displays, such as the Nokia
N810 tablet having a resolution of 800x600 pixels.

FlightDeck on the other hand requires at least 1024x768 screen
resolution. The remainder of this document describes the FlightDeck
application. Despite the package name, it is also suitable for IFR
flights.

\subsection{User Interface}

FlightDeck features a user interface similar to state of the art glass
cockpit avionics. Its user interface is organized into a set of
pages. At the bottom of the screen, there are 12 ``soft keys'', which
map to the function keys F1 to F12 (but can of course also be
clicked). The first five soft keys activate a page in a page
group. Subsequent presses of the same soft key advances to the next
page in the same page group. The page groups are:

\begin{itemize}
\item NAV -- Main Navigation Pages
  \begin{itemize}
  \item NAV:M -- Map Page
  \item NAV:AD -- Airport Map Page
  \end{itemize}
\item WPT -- Waypoint Information
  \begin{itemize}
  \item WPT:ARPT -- Airport Information Page
  \item WPT:NAV -- Navaid Information Page
  \item WPT:INT -- Intersection Information Page
  \item WPT:AWY -- Airway Segment Information Page
  \item WPT:ASPC -- Airspace Information Page
  \item WPT:MAP -- Map Element Information Page
  \item WPT:FPL -- Flight Plan Waypoint Information Page
  \end{itemize}
\item FPL -- Flight Plans
  \begin{itemize}
  \item FPL -- Current Flight Plan Page
  \item FPLS -- Stored Flight Plan Directory Page
  \item FPL -- Avionics GPS Flight Plan Import Page
  \end{itemize}
\item AUX -- Auxilliary
  \begin{itemize}
  \item AUX -- Document Directory Page
  \item AUX -- Document Page
  \item AUX -- Performance Calculator Page
  \item AUX -- Current Flight Plan Sunrise Sunset Page
  \end{itemize}
\item SENS -- Sensor Configuration
  \begin{itemize}
  \item SENS:LOG -- System Log Page
  \item SENS:X -- Sensor Configuration Pages
  \end{itemize}
\end{itemize}

Built-in keyboard widgets allow FlightDeck to be operated without
keyboard solely with a pen, a mouse or by tapping on touchscreen
enabled devices.

\subsection{Sensors}

FlightDeck supports the connection of various sensors:

\begin{itemize}
\item Position Sensors (GPS devices) supported by gpsd
\item Avionics GPS devices outputting King format
\item liblocation supported position sensors
\item PlayStation Move controller as attitude and heading sensor
\item MS5534 barometric sensor
\end{itemize}

Multiple sensors of the same kind, for example multiple GPS devices,
are supported. User selectable priorities specify which sensor is used
for what type of information under what circumstances.

\subsection{Databases and Configuration Files}

FlightDeck requires map and aviation information databases. Global
databases are stored under {\tt /usr/share/vfrnav}. Per user
modifications to the global databases (record deletes and additions)
are stored under {\tt \$(HOME)/.vfrnav}.

Databases may be created from various sources and file formats;
conversion utilities are included. Alternatively, worldwide coverage
database files may be downloaded from
{\tt http://www.baycom.org/$\sim$tom/vfrnav/data/repoview/}. Unfortunately,
there is no accurate, up-to-date, official publication; the accuracy
and actuality of the information contained in the databases cannot be
guaranteed -- the user is required to check that the information meets
his requirements.

\begin{figure}[!htbp]
  \begin{center}
    \jnxfig[scale=0.5]{splashdialog.eps}{splashdialog.png}
    \caption{Splash Screen}
    \label{fig:splashdialog}
  \end{center}
\end{figure}

FlightDeck supports multiple configurations; on start-up, FlightDeck
searches for available configurations and presents the user with a
list of the configurations found. The user can then select one that
will be used. Figure \ref{fig:splashdialog} shows the configuration
selection dialog.

Global configurations are stored under {\tt /usr/share/vfrnav/flightdeck},
per user configurations are stored under {\tt \$(HOME)/.vfrnav/flightdeck}.
Per user configurations take precedence over global
configurations. Configuration files use a simple name=value ascii
format, with section names in square brackets. They should use a file
name of the form {\tt {\it xx}.cfg}.

Configurations store the sensor setup, as well as user settings so
that the next run with the same configuration continues with the same
settings as the last run ended.

Each configuration should reference an aircraft definition
file. Aircraft definition files are stored under
{\tt \$(HOME)/.vfrnav/aircraft}. They describe the aircraft and
contain Weight \& Balance and Performance tables.

\clearpage

\section{Navigation Page Group}

\begin{figure}[!htbp]
  \begin{center}
    \jnxfig[scale=0.5]{navmap.eps}{navmap.png}
    \caption{Nav Page: Map}
    \label{fig:navmap}
  \end{center}
\end{figure}

\begin{figure}[!htbp]
  \begin{center}
    \jnxfig[scale=0.5]{navarpt.eps}{navarpt.png}
    \caption{Nav Page: Airport Map}
    \label{fig:navarpt}
  \end{center}
\end{figure}

The main navigation page (Figure \ref{fig:navmap}) consists of five
parts: the soft keys at the bottom, the vector map with an aircraft
symbol occupying the largest space on the left, an attitude indicator
on the top left, an altitude indicator and flight plan informations in
the middle on the left side, and a horizontal situation indicator on
the bottom left.

Clicking on the NAV soft key again switches to the airport map
page. This page looks the same, except that the vector map is replaced
by an airport map (if airport map data is available).

\subsection{Attitude Indicator}

The Attitude Indicator in the top left edge works like a standard
Attitude Indicator, displaying roll and pitch angles. It only works if
an attitude sensor is connected, such as the PlayStation Move.

A trapezoid below the orange sky pointer indicates slip.

\subsection{Flight Plan Information}

Flight Plan information is displayed on the center left edge. Distance
(in nmi), magnetic track, vertical speed, desired track and time to
the next flight plan or direct-to waypoint are displayed first,
followed by distance, magnetic track, vertical speed and time to the
destination.

WPT TIME counts the time since passing the last waypoint, flight time
counts the time since departure. TIMER is a generic up-counting timer
that can be started and stopped using a soft key.

IAS and TAS display entered or calculated air speeds, and F, FF and \%
display fuel consumed since start, calculated fuel flow (from the
aircraft model), and calculated engine power relative to maximum rated
engine power.

\subsection{Altitude Indicator}

The central element of the Altitude Indicator is the altitude
tape. The vertical speed indicator is located on the right side of the
altitude tape. A magenta diamond indicates the vertical speed required
to meet the next waypoint altitude target.

The altitude bug altitude is displayed above the tape. The current
altimeter setting (or STD, for standard) is displayed below the
tape. The white number on brown background in the bottom right edge
indicates terrain altitude at the current position. Brown background
inside the altitude tape also indicates terrain altitude (and below).

A yellow chevron at the left edge of the tape indicates the desired
altitude to reach the next waypoint altitude target (similar to a
``glide slope''). A red chevron indicates the configured minimum.

Since altitude is an important parameter (especially for IFR traffic),
and most aircraft have many different altitude sensors, white labels
may be displayed at the left edge of the tape for each altitude
source, even if it was not selected to provide the system
altitude. This allows the pilot to monitor the accuracy of all
altitude sources.

Clicking into the altitude indicator opens the Altitude Configuration
Dialog (Figure \ref{fig:altcfgdlg}).

\subsection{Horizontal Situation Indicator}

The HSI displays compass and horizontal navigation information. The
current heading is displayed in orange at the top if magnetic heading
information is available (eg. from a connected PlayStation Move
sensor). If no heading sensor is available, the heading may be entered
manually, in which case it will be displayed prefixed with an M. In
the absence of any heading information, the current course is
displayed, prefixed with a C. If heading information is available, the
course is displayed as a magenta diamond on the periphery of the
compass rose.

A magenta arc at the top periphery of the compass rose indicates turn
rate. An arrow head indicates that the turn rate is larger than
approximately one and a half of a standard rate turn, otherwise the
end of the arc indicates the heading 6 seconds into the future.

The yellow pointer indicates the desired track to the next waypoint,
as well as the deviation from it.

An optional second cyan pointer may be set to point to any waypoint.

The top left corner displays where the pointer(s) point to.

The top right corner displays air data, and the magnetic variation at
the present position.

The bottom right corner displays wind information, if heading
information and air data is available.

Terrain information may be optionally displayed in the background of
the HSI. Terrain graphics is normally black. Terrain that is less than
2000ft below the current altitude is displayed in green, terrain less
than 1000ft below is displayed in yellow, terrain less than 500ft
below is displayed in light orange, terrain less than 200ft below is
displayed in dark orange, and terrain above the current altitude is
displayed in red. The current position is the center of the compass
rose.

Optionally, the terrain graphics may also display the current flight
plan, airports, navaids and intersections.

Furthermore, the locations reachable from the current position and
altitude in unpowered glide may be displayed in gray. The glide area
graphics considers aircraft performance (glide ratio), wind (the wind
manually entered in the flight plan page, \emph{not} the automatically
calculated wind), and topography, giving the pilot a good idea which
emergency landing locations may be feasible.

There are some inaccuracies in the computation the pilot needs to
consider:

\begin{itemize}
\item Best glide speed is taken as TAS, while for most aircraft it is
  IAS -- only relevant for wind corrections
\item The available terrain information is relatively coarse
\item The algorithm does not enforce any minimum height for crossing
  mountain ridges
\end{itemize}

Clicking into the HSI opens the HSI Configuration Dialog (Figure
\ref{fig:hsicfgdlg}).

\subsection{Soft Keys}

The following functions are available on the soft keys:

\begin{description}
\item[D:NORM] Map Declutter; reduces the information displayed on the
  vector map
  \begin{description}
  \item[D:NORM] Display user selected information
  \item[D:ASPC] Display User selected information minus Topography,
    Terrain and Airways
  \item[D:AWY] Display User selected information minus Topography,
    Terrain and Airspaces
  \item[D:TERR] Display User selected information minus Airways and
    Airspaces
  \end{description}
\item[TIMER] Starts and Stops an utility up-counting timer (for timed
  IFR approach segments, or holdings)
\item[FPLNAV] Enable or Disable Flight Plan Navigation. When enabling,
  it selects Direct To the nearest waypoint of the actual flight plan
\item[HOLD] Enables or Disables Hold mode. In Hold mode, waypoint
  sequencing is stopped, and guidance is provided inbound the next
  waypoint on a selected radial. Enabling Hold pops up the hold dialog
  (Figure \ref{fig:holddialog}), allowing the user to select the
  holding inbound track.
\item[BRG2] Enables or Disables the second (cyan) bearing pointer in
  the HSI. Note that before enabling the bearing pointer, it should be
  set to point to a waypoint either in one of the waypoint pages or
  the flight plan page.
\item[SETUP] Activates the Setup soft keys.
\end{description}

\begin{figure}[!htbp]
  \begin{center}
    \jnxfig[scale=0.5]{holddialog.eps}{holddialog.png}
    \caption{Hold Dialog}
    \label{fig:holddialog}
  \end{center}
\end{figure}

\subsection{Setup Soft Keys}

The Setup soft keys provide the following functions:

\begin{description}
\item[MAP$+$] Zoom the vector map in
\item[MAP$-$] Zoom the vector map out
\item[ARPT$+$] Zoom the airport map in
\item[ARPT$-$] Zoom the airport map out
\item[TERRAIN$+$] Zoom the terrain map underlying the HSI in
\item[TERRAIN$-$] Zoom the terrain map underlying the HSI out
\item[MAPCFG] Open the Map Configuration Dialog (Figure \ref{fig:mapcfgdlg})
\item[ALTCFG] Open the Altitude Configuration Dialog (Figure \ref{fig:altcfgdlg})
\item[HSICFG] Open the HSI Configuration Dialog (Figure \ref{fig:hsicfgdlg})
\item[WINDOW/FULLSCREEN] Toggles between Windowed and Fullscreen
  (borderless) Mode
\item[QUIT] Terminates the FlightDeck Application
\item[BACK] Return to the Main Soft Keys
\end{description}

\subsection{Altitude Configuration Dialog}

\begin{figure}[!htbp]
  \begin{center}
    \jnxfig[scale=0.5]{altcfgdlg.eps}{altcfgdlg.png}
    \caption{Altitude Configuration Dialog}
    \label{fig:altcfgdlg}
  \end{center}
\end{figure}

Figure \ref{fig:altcfgdlg} shows the altitude configuration dialog.

\begin{description}
\item[Altitude Bug] Enable/Disable the altitude bug, and enter its altitude
\item[Minimum] Enable/Disable the minimum indicator, and enter its altitude
\item[QNH] Current Altimeter Setting
\item[QNH/Standard] Switch between the current Altimeter Setting
  (``Altitudes'') and Standard (``Flight Level'')
\item[hPa/inHg] Switch between Hectopascals and Inches of Mercury as
  Altimeter Setting Unit
\item[Altitude Labels] Enable/Disable Altitude Labels, i.e. display of
  Altitudes of non-selected Altitude Sources
\item[Metric] Switch between Metric (``m'') and Imperial (``ft'') units
\end{description}

{\bf Important:} Do \emph{not} just enter 1013 or 29.92 into the QNH
field when switching to standard, but use the Standard switch, because
the current QNH is needed by FlightDeck even in standard mode to
compute barometric altitudes from true altitude sources, such as GPS.

\subsection{HSI Configuration Dialog}

\begin{figure}[!htbp]
  \begin{center}
    \jnxfig[scale=0.5]{hsicfgdlg.eps}{hsicfgdlg.png}
    \caption{HSI Configuration Dialog}
    \label{fig:hsicfgdlg}
  \end{center}
\end{figure}

Figure \ref{fig:hsicfgdlg} shows the HSI configuration dialog.

\begin{description}
\item[Manual Heading] Enable/Disable the manual heading, and enter the
  current heading; manual heading may be magnetic or true
\item[RAT] Enter the Ram Air Temperature (i.e. what the thermometer
  reads); using TAS and current altitude, FlightDeck computes the
  current outside air temperature (OAT). When altitude changes, FlightDeck
  uses the ISA lapse rate to adjust the outside air temperature.
\item[IAS] Select and Enter Indicated Airspeed. What needs to be
  entered is actually Calibrated Airspeed, FlightDeck currently cannot
  correct for Instrument Errors. If selected, TAS is computed from the
  given IAS value, and altitude.
\item[Cruise RPM/MP] Enter cruise RPM and Manifold Pressure (MP)
  Settings. If selected, the TAS is computed from Aircraft performance
  data, given the current Density Altitude, and the selected RPM/MP
  values. If the current vertical speed is at least 150ft/min
  climbing, $v_Y$ is used instead.
\item[Display Wind] whether the calculated wind (given TAS and
  Heading) should be displayed
\item[Display Flight Plan] whether the current flight plan should be
  displayed below the compass rose
\item[Display Flight Plan Labels] whether flight plan waypoints should
  have labels
\item[Display Terrain] whether terrain map should be displayed below
  the compass rose
\item[Display Airports] whether airports should be displayed
\item[Display Navaids] whether navaids should be displayed
\item[Display Waypoints] whether intersections should be displayed
\item[Display Glide] whether places reachable by unpowered glide
  shoudl be displayed
\end{description}



\subsection{Map Configuration Dialog}

\begin{figure}[!htbp]
  \begin{center}
    \jnxfig[scale=0.5]{mapcfgdlg.eps}{mapcfgdlg.png}
    \caption{Map Configuration Dialog}
    \label{fig:mapcfgdlg}
  \end{center}
\end{figure}

Figure \ref{fig:mapcfgdlg} shows the map configuration dialog.

This dialog configures what information should be displayed on the
vector map.

\clearpage

\section{Waypoint Page Group}

\begin{figure}[!htbp]
  \begin{center}
    \jnxfig[scale=0.5]{wptarpt1.eps}{wptarpt1.png}
    \caption{Waypoint Page: Airport}
    \label{fig:wptarpt1}
  \end{center}
\end{figure}

The waypoint page group displays information about database
elements. Database elements may be searched by entering text, nearest
to the current position (when entered using the WPT soft key) or
nearest to the currently selected flight plan waypoint (when entered
by clicking on insert in the flight plan page). Distance values are
updated if the current position changes.

Selected database elements may be used as direct-to target, or as
target for the second HSI pointer (BRG2).

The waypoint pages consist of 3 major parts: the top half of the
screen is used by the element list, the bottom left half displays the
element on the vector map, and the bottom right half displays textual
information about the selected element.

\begin{figure}[!htbp]
  \begin{center}
    \jnxfig[scale=0.5]{wptarpt2.eps}{wptarpt2.png}
    \caption{Waypoint Page: Airport with Airport Map}
    \label{fig:wptarpt2}
  \end{center}
\end{figure}

Figure \ref{fig:wptarpt1} shows the waypoint airport information page
with the vector map displayed, while Figure \ref{fig:wptnavaid} shows
the same page with the airport map.

If the airport database contains approach or departure procedures,
they may be displayed by clicking on the expander next to the airport
name.

Airports that are within glide distance (\emph{without considering
  terrain}, but considering wind entered on the flight plan page) are
displayed in italics.

Airports that have no suitable runway (considering entered mass from
the Weight \& Balance page, and wind entered on the flight plan page)
are coloured red. Airports whose ``best'' runway only allows either
landing or takeoff, but not both, are coloured orange. The same colour
scheme is applied to the runway list as well.

\begin{figure}[!htbp]
  \begin{center}
    \jnxfig[scale=0.5]{wptnavaid.eps}{wptnavaid.png}
    \caption{Waypoint Page: Navaid}
    \label{fig:wptnavaid}
  \end{center}
\end{figure}

Figure \ref{fig:wptnavaid} shows the waypoint navaid information
page.

\begin{figure}[!htbp]
  \begin{center}
    \jnxfig[scale=0.5]{wptint.eps}{wptint.png}
    \caption{Waypoint Page: Intersection}
    \label{fig:wptint}
  \end{center}
\end{figure}

Figure \ref{fig:wptint} shows the waypoint intersection information
page.

\begin{figure}[!htbp]
  \begin{center}
    \jnxfig[scale=0.5]{wptmapel.eps}{wptmapel.png}
    \caption{Waypoint Page: Map Element}
    \label{fig:wptmapel}
  \end{center}
\end{figure}

Figure \ref{fig:wptmapel} shows the waypoint map elment information
page. Map elements include cities and lakes.

\paragraph{Soft Keys}

\begin{description}
\item[D:NORM] Declutter Mode, same as in the main menu
\item[ZOOM$+$] Zoom Map In
\item[ZOOM$-$] Zoom Map Out
\item[MAP] Change Map Mode (cycle through Vector Map, Terrain and
  Airport Map)
\end{description}

\clearpage

\section{Flight Plan Page Group}

\begin{figure}[!htbp]
  \begin{center}
    \jnxfig[scale=0.5]{fplbrowser.eps}{fplbrowser.png}
    \caption{Flight Plan Browser Page}
    \label{fig:fplbrowser}
  \end{center}
\end{figure}

Figure \ref{fig:fplbrowser} shows the Flight Plan Database Browser
Page. It may be used to create a new flight plan, to load, delete,
duplicate or reverse an existing flight plan.

To create a new flight plan, first select New FPL, then load the newly
created empty flight plan.

\subsection{Flight Plan Page}

\begin{figure}[!htbp]
  \begin{center}
    \jnxfig[scale=0.5]{fpleditor.eps}{fpleditor.png}
    \caption{Flight Plan Editor Page}
    \label{fig:fpleditor}
  \end{center}
\end{figure}

Figure \ref{fig:fpleditor} shows the flight plan editor. The top half
displays the curent flight plan. The bottom left half displays the
currently selected waypoint on a map. The bottom right half displays
and allows to edit textual information about the currently selected
waypoint.

\paragraph{Soft Keys}

\begin{description}
\item[D:NORM] Declutter Mode, same as in the main menu
\item[ZOOM$+$] Zoom Map In
\item[ZOOM$-$] Zoom Map Out
\item[SCRKBD] Enable/Disable the on-screen keyboard (for touchscreen
  devices; Figure \ref{fig:screenkbd})
\item[MAP] Change Map Mode (cycle through Vector Map, Terrain and
  Airport Map)
\end{description}

\begin{figure}[!htbp]
  \begin{center}
    \jnxfig[scale=0.5]{screenkbd.eps}{screenkbd.png}
    \caption{On Screen Keyboard}
    \label{fig:screenkbd}
  \end{center}
\end{figure}

\paragraph{Buttons}

\begin{figure}[!htbp]
  \begin{center}
    \jnxfig[scale=0.5]{fpltextentry.eps}{fpltextentry.png}
    \caption{Flight Plan Text Entry}
    \label{fig:fpltextentry}
  \end{center}
\end{figure}

\begin{description}
\item[Insert WPT] Duplicates the currently selected Waypoint, and
  opens the Waypoint Page in nearest mode, allowing to select a
  waypoint from the database
\item[Insert Txt] Opens the Flight Plan Text Entry Dialog (Figure
  \ref{fig:fpltextentry}), which allows multiple waypoints to be
  entered in ``flight plan format'', for example: ``LSZK BARIG J51 WIL
  BINGI ARVAN LSZG''. Points may include airports, intersections and
  navaids, or coordinates. Airways are also acceptable. It is not
  necessary to enter airway entry and exit points; if missing, the
  nearest point on the airway is searched and entered as well. This is
  usually the fastest method to enter a flight plan; some clean-up may
  be necessary though (mostly deleting unnecessary waypoints).
\item[Duplicate WPT] Duplicate the currently selected Waypoint
\item[Move WPT] Move the currently selected Waypoint up or down
\item[Delete WPT] Delete the currently selected Waypoint
\item[Straighten WPT] Shifts the currently selected Waypoint such that
  it lies on a straight line from the previous to the next Waypoint
\item[Direct To] Opens the Join dialog (Figure \ref{fig:coordeditor}),
  which allows the user to select either Direct To the currently
  selected Waypoint, or Join the Leg between the previous and the
  currently selected Waypoint, and then start flight plan navigation
\item[BRG 2] Point the second (cyan) HSI pointer to the currently
  selected Waypoint
\end{description}

Waypoint information may be edited directly in the lower right part of
the page. Clicking on the Edit button opens the Coordinate Editor
dialog (Figure \ref{fig:coordeditor}), which offers additional options
for entering coordinates (such as radial/distance from the current
point, or even by clicking on the map).

\begin{figure}[!htbp]
  \begin{center}
    \jnxfig[scale=0.5]{coordeditor.eps}{coordeditor.png}
    \caption{Coordinate Editor}
    \label{fig:coordeditor}
  \end{center}
\end{figure}

\section{Auxillary Page Group}

The Auxillary Page Group contains an embedded document (PDF) reader
suitable for quickly displaying approach charts, a performance
calculator and sunrise/sunset information for the current flight plan.

\subsection{Document Reader}

\begin{figure}[!htbp]
  \begin{center}
    \jnxfig[scale=0.5]{docdirpage.eps}{docdirpage.png}
    \caption{Documents Directory Page}
    \label{fig:docdirpage}
  \end{center}
\end{figure}

Figure \ref{fig:docdirpage} shows the documents directory. FlightDeck
searches user-selectable directories for displayable documents. It
analyzes the file name of each document found, and determines which
files may be relevant to each current flight plan
waypoint. Furthermore, all documents found are also listed under the
``all documents'' heading.

Selecting a document and clicking on the AUX soft key leads to the
document page.

\begin{figure}[!htbp]
  \begin{center}
    \jnxfig[scale=0.5]{docpage.eps}{docpage.png}
    \caption{Document Page}
    \label{fig:docpage}
  \end{center}
\end{figure}

Figure \ref{fig:docpage} shows the document page.

\paragraph{Soft Keys}

\begin{description}
\item[FIT] Toggles between Best Fit and Fit Width mode
\item[ZOOM$+$] Zoom Document In
\item[ZOOM$-$] Zoom Document Out
\item[ROTATE] Rotate Document by 90 degrees
\item[DOC$+$] Advance to the next document in the list
\item[DOC$-$] Return to the previous document in the list
\item[BACK] Return to the Documents Directory Page
\end{description}

\subsection{Performance Calculator}

\begin{figure}[!htbp]
  \begin{center}
    \jnxfig[scale=0.5]{perfpage.eps}{perfpage.png}
    \caption{Performance Calculator Page}
    \label{fig:perfpage}
  \end{center}
\end{figure}

Figure \ref{fig:perfpage} shows the performance calculator page. The
left side computes takeoff and landing distances, while the right side
computes the weight \& balance graph.

Clicking on ``From W\&B'' copies the masses from the Weight \& Balance
graph. The Takeoff mass is directly set from the Weight \& Balance
total mass, while the Landing mass has the total fuel (from the
currently loaded flight plan) subtracted (but not more than sum total
of the entered fuel in the Weight \& Balance calculation).

Clicking on ``From FPL'' copies the takeoff and landing airfield
elevations from the currently loaded Flight Plan.

\subsection{Sunrise/Sunset Calculator}

\begin{figure}[!htbp]
  \begin{center}
    \jnxfig[scale=0.5]{srsspage.eps}{srsspage.png}
    \caption{ Page}
    \label{fig:srsspage}
  \end{center}
\end{figure}

Figure \ref{fig:srsspage} shows the Sunrise/Sunset calculator. The
user may select a date, for which the calculator displays sunrise and
sunset times (in UTC) for each waypoint of the currently loaded Flight
Plan. Note that the calculator considers only geographical positions
of the waypoints; the official day/night times may differ slightly due
to country rules.

\section{Sensors Page Group}

\begin{figure}[!htbp]
  \begin{center}
    \jnxfig[scale=0.5]{logpage.eps}{logpage.png}
    \caption{System Log Page}
    \label{fig:logpage}
  \end{center}
\end{figure}

Figure \ref{fig:logpage} shows the system log page.



\raggedright
\bibliography{attitude,propeller}
\bibliographystyle{base/plain}

\appendix
\section{Known Limitations}

\subsection{Windows Version}

\begin{itemize}
\item Document Viewer is not available
\item Only very limited I/O drivers available
\item Lightly Tested
\end{itemize}


\section{Technical}

\subsection{IFR Autorouter}

\subsection{Introduction}

IFR Flight planning in countries such as the United States is fairly
simple. Given a RNAV equipped aircraft, flight plans between arbitrary
waypoints may be filed with little restrictions.

Not so in the ECAC\footnote{European Civil Aviation Conference}
region. An IFR flight plan must comply with many rules (more than 6000
early 2013) before it is accepted. When submitted to
CFMU\footnote{Central Flow Management Unit} in Bretigny or Brussels, a
computer checks the flight plan and rejects it if any one of those
rules is violated. IFR Flight Planning is even more complicated by the
fact that the ECAC region is comprised by more than 20 states, each
with sometimes significantly different rules.

\noindent\begin{figure}[!htbp]
  \begin{center}
    \jnxfig[scale=0.7]{routenetwork.eps}{routenetwork.pdf}
    \caption{Central Switzerland Airway Network}
    \label{fig:routenetwork}
  \end{center}
\end{figure}

IFR Flight Plans generally must follow the route network. Figure
\ref{fig:routenetwork} shows an exceprt of the central Switzerland
route network. The route network consists of Navigation Aids and
Intersections. Intersections are specific coordinates that were given
a 5 character identifier. Navaids and Intersections are connected by
airway segment. An airway also has a name -- for example G5 between
WIL and FRI. Airways are often limited vertically. Furthermore, they
may be unidirectional.

An IFR Flight Plan generally joins the route network at a point close
to the departure aerodrome and leave the route network at a point
close to the destination aerodrome. There are rules that govern which
joining and leaving points are allowed.

Clearly, the complexity of finding an IFR Flight Plan that honours all
rules, but also limitations of the aircraft, has become impractical to
do by hand, especially for low flying aircraft and for the General
Aviation community, where a Flight Plan is used only once. Flight
Planning can easily exceed the flight time even for slow
aircraft. There is therefore no way around automating construction of
an acceptable IFR flight plan.

An autorouter should have the following desirable properties:
\begin{enumerate}
\item Minimal user input; ideally it should only require the departure
  and destination aerodrome, and the aircraft to be flown
\item Additional altitude restrictions, for example due to weather (icing)
\item Find the optimal route between the given aerodromes for the
  given aircraft; optimality criteria should include flight time and
  fuel consumption and be user selectable
\end{enumerate}

The router descibed in this document achieves these
properties. Additionally, it would be desirable to have the following
additional properties:
\begin{enumerate}
\item Offline operation
\item Consider winds aloft
\end{enumerate}

Offline operation is currently infeasible, unfortunately, as the rules
are apparently not available completely in an electronically useable
format.

Winds aloft awareness should be easy to add to the general algorithm
structure.

The autorouter algorithm performs the following steps:

\begin{enumerate}
\item Construct Routing Graph
  \begin{enumerate}
  \item Add Navaid and Intersection vertices
  \item Add Airway edges
  \item Remove Intersections with non-flightplannable names
  \item Add Direct-To (DCT) edges
  \item Add Departure and Destination Aerodromes, and SID and STAR edges
  \end{enumerate}
\item Router Iteration
  \begin{enumerate}
  \item Find the (k-)shortest path in the Routing Graph from Departure
    to Destination
  \item Check the path against the locally available rules
  \item If the local check passes, submit it to the CFMU flight plan
    validation website
  \item If it passes the CFMU test as well, we are done
  \item (Try to) interpret the error messages; if an edge is clearly
    forbidden for all possible flight plans between these aerodromes,
    remove the edge from the Routing Graph and compute the shortest
    path during the next iteration, otherwise compute the next
    shortest path during the next iteration
  \end{enumerate}
\end{enumerate}

\subsection{Routing Graph Construction}

Operations research provides us with a great many publications on
algorithms for finding paths in graphs.

A graph consists of vertices and edges. Each edge connects two
vertices. A directed graph (digraph) is a graph with directed edges
(i.e. each edge may only be travelled in one direction). Parallel
edges are allowed -- there may be multiple airways between two
intersections. Each edge has a ``weight'' -- for example the distance,
flight time, or fuel consumption when crossing the edge. A path
through the graph from one vertex to another vertex may only follow
edges; the weight of the path is the sum of the weights of the edges
crossed. There are efficient algorithms for computing the shortest
path (minimum weight path) in a graph, for example the Dijkstra
algorithm. Unfortunately, computing the second best or $k$th best path
is significantly harder (for example the Yen algorithm).

The digraph concept matches the airway network well. Navaids and
Intersections correspond to vertices, and airway segments to edges.

What weight should the edges be given? Distance might be chosen, but
usually, the user is not interested in the shortest route, as the
shortest route may require a high flight level and a long climb. A
slightly longer route on a lower flight level may be faster,
especially on aircraft equipped with normally aspirated engines. So
the weight is set to either the travel time on the leg, or the fuel
consumption, depending on user selection.

What altitude should be chosen for a flight plan? Often, no compliant
route exists between two aerodromes on a single flight level, or large
detours would have to be flown. Therefore, the autorouter should be
able to handle level changes on every flight plan leg. On the other
hand, the user should not need to select the altitude; the router
should automatically select the best altitude from an aircraft model
and the terrain. It should be possible, however, to restrict the
altitude range, for example because of icing. So how can flight levels
be mapped to the graph concept?

\noindent\begin{figure}[!htbp]
  \begin{center}
    \jnxfig[scale=0.7]{vertex.eps}{vertex.pdf}
    \caption{Routing Graph}
    \label{fig:vertex}
  \end{center}
\end{figure}

Figure \ref{fig:vertex} shows the solution. Consider flight
planning on four levels from FL70 to FL100, for simplicity. Navaids
and Intersections map to vertices.

Each airway edge carries not just one ``weight'', but a table of
``weights'' for each flight level, in this case travel time in
seconds. Figure \ref{fig:vertex} also shows level and
directionality constraints, see N491 which only exists at FL090 and
higher, and only in the direction TRA$\rightarrow$ZUE. Nonexisting
levels are marked with NaN (not a number).

Level changes are handled by using a modified version of the Dijkstra
algorithm, that does not just keep track of vertices, but also of
levels. If the algorithm tries a level change, it adjusts the edge
weight. For climbs, it adds the additional time or fuel needed for the
climb. This figure is computed using the aircraft performance
model. In descent, no adjustment is made. It could theoretically be
negative; however, in my experience, in practice, I am seldom allowed
to fly an optimal descent towards the destination, I almost always
arrive at close to the cruise level in the vincinity of the
destination, and then need additional track miles for the
descent. Therefore, I have chosen to not give credits for descent.

Increasingly, airways contain points that are not flight-plannable
(i.e. may not be submitted in a flight plan). An example above is
BER10 between BERSU and URIGI. These points need to be removed; all
edges incident on such vertices need to be replaced by edges between
predecessor and successor vertices, with the weight set to the sum
of the weight of the inbound and the outbound edge. Only pairs of
inbound and outbound edges belonging to the same flight level and the
same airway should be considered.

Besides airway segments, direct routes may be allowed under some
conditions, for example BERSU to WIL in the graph above. The
conditions where direct-to segments (DCT) are allowed differ greatly
from country to country. Switzerland and Austria are very restrictive,
in Germany and parts of France they are allowed for low level IFR up
to a certain leg length, while in Poland it is often the only
possibility due to the thin low level route network. Clearly, the
router has to support direct-to segments.

DCT edges are added on all (super-)vertex pairs up to a configurable
maximum. DCT edges are not added if an airway route that is not
significantly longer (currently 1\%) than the DCT is
available. Furthermore, DCT edges are only added for altitudes 1000ft
above terrain (or 2000ft if the terrain is higher than
5000ft). Terrain altitudes are computed in a corridor $\pm$5 nautical
miles around the center line.

Finally, the departure and destination aerodrome vertex needs to be
added. These are simple vertices, as the aerodrome has a fixed
altitude. From the departure aerodrome vertex, an SID edge is added to
every flight level of every possible entry point. Similarily, STAR
edges are added to the destination aerodrome vertex. Entry and exit
points may either be specified, or every point with a maximum
configurable distance is tried. Since procedure databases with exact
routings are not available to me and the actual runway used is not
known at flight planning time, the direct distance between the
aerodrome and the entry/exit point is computed. The weight is then the
time or fuel needed to travel that distance at the flight level at the
entry/exit point. For SID edges, the weight is increased by the
additional time or fuel needed to climb from aerodrome elevation to
the given flight level; for STAR edges, the weight is not reduced by
the descent, for the reasons given above.

Due to the flight plan format and the way the CFMU checker works, the
flight level must not be changed immediately after the SID or before
the STAR. This is enforced by duplicating the entry and exit
supervertex without level change edges, and duplicating all other
incident edges.



\subsection{Waypoint Sequencing}

Waypoint sequencing switches to the next waypoint whenever it
determines that the turn must be initiated to smoothly reach the next
track (it assumes all waypoints are fly-by waypoints at the moment!).

\noindent\begin{figure}[!htbp]
  \begin{center}
    \jnxfig[scale=0.7]{wptseq.eps}{wptseq.pdf}
    \caption{Waypoint Sequencing Geometry}
    \label{fig:wptseq}
  \end{center}
\end{figure}

$\alpha=\mathrm{Outbound Track}-\mathrm{Inbound Radial}$ is the angle
describing the course change.

$r=\frac{v\cdot{}2\mathrm{min}}{2\pi}$ is the radius of a standard
rate turn.

$d=\frac{r}{\tan{\frac{\alpha}{2}}}$ is the distance from the waypoint
where the turn must be initiated.

$\tau=(\frac{\pi}{2}-\frac{\alpha}{2})\cdot{}2\cdot{}\frac{r}{v}=(\pi-\alpha)\frac{r}{v}=(\pi-\alpha)\frac{2\mathrm{min}}{2\pi}=(1-\frac\alpha\pi)1\mathrm{min}$
is the time needed to fly half the arc between inbound radial and outbound track.

\subsection{MS5534 Interface}

\begin{figure}[!htbp]
  \begin{center}
    \jnxfig[angle=0,width=0.8\linewidth]{pbxif.eps}{pbxif.pdf}
    \caption{MS5534 Interface}
    \label{fig:pbxif}
  \end{center}
\end{figure}
\clearpage

\subsection{Weather Charts}

\subsubsection{Barycentric Interpolation}

\begin{figure}[!htbp]
  \begin{center}
    \jnxfig[angle=0,scale=1]{barycentric1.eps}{barycentric1.pdf}
    \caption{Barycentric Coordinates in a Triangle}
    \label{fig:barycentric1}
  \end{center}
\end{figure}

\begin{figure}[!htbp]
  \begin{center}
    \jnxfig[angle=0,scale=1]{barycentric2.eps}{barycentric2.pdf}
    \caption{Barycentric Coordinates in a Triangle}
    \label{fig:barycentric2}
  \end{center}
\end{figure}

The barycentric coordinates are given by
\begin{equation}
  (a_0+a_1+a_2)\cdot\vec{p} = a_0\cdot\vec{x_0}+a_1\cdot\vec{x_1}+a_2\cdot\vec{x_2}
\end{equation}

Additionally, $a_0+a_1+a_2=1$ to make them unique.

The ratio between barycentric coordinates is the same as the ratio of
the correspondingly coloured triangles.

The interpolated value for $\vec{p}$ is the sum of the triangle vertex
values weighted by their correspondic barycentric coordinate.

In order to interpolate within a rectangle, the rectangle is
decomposed into four triangles each consisting of the center point
(whose value is the average of all four corner points) and two of the
corner points.

Assume $x_0=x_2$, $x_1=x_3$, $y_0=y_1$ and $y_2=y_3$.

Set:
\begin{equation}
  a_3=\frac{x_p-x_0}{x_1-x_0}\frac{y_p-y_0}{y_2-y_0}
\end{equation}

\begin{equation}
  a_2=\frac{x_p-x_1}{x_0-x_1}\frac{y_p-y_0}{y_2-y_0}
\end{equation}

\begin{equation}
  a_1=\frac{x_p-x_0}{x_1-x_0}\frac{y_p-y_2}{y_0-y_2}
\end{equation}

\begin{equation}
  a_0=\frac{x_p-x_1}{x_0-x_1}\frac{y_p-y_2}{y_0-y_2}
\end{equation}


\begin{equation}
  \begin{split}
    \sum_{i=0}^{3}a_i &= \frac{x_p-x_0}{x_1-x_0}\frac{y_p-y_0}{y_2-y_0}+
    \frac{x_p-x_1}{x_0-x_1}\frac{y_p-y_0}{y_2-y_0} +
    \frac{x_p-x_0}{x_1-x_0}\frac{y_p-y_2}{y_0-y_2} +
    \frac{x_p-x_1}{x_0-x_1}\frac{y_p-y_2}{y_0-y_2} \\
    &= \frac{x_p-x_0-x_p+x_1}{x_1-x_0}\frac{y_p-y_0}{y_2-y_0}+
    \frac{x_p-x_0-x_p+x_1}{x_1-x_0}\frac{y_p-y_2}{y_0-y_2} \\
    &= \frac{y_p-y_0}{y_2-y_0}+\frac{y_p-y_2}{y_0-y_2}
    = \frac{y_p-y_0-y_p+y_2}{y_2-y_0} = 1
  \end{split}
\end{equation}

\begin{equation}
  \begin{split}
    \sum_{i=0}^{3}a_ix_i &= \frac{x_p-x_1}{x_0-x_1}\frac{y_p-y_2}{y_0-y_2}x_0+
    \frac{x_p-x_0}{x_1-x_0}\frac{y_p-y_2}{y_0-y_2}x_1+
    \frac{x_p-x_1}{x_0-x_1}\frac{y_p-y_0}{y_2-y_0}x_0+
    \frac{x_p-x_0}{x_1-x_0}\frac{y_p-y_0}{y_2-y_0}x_1 \\
    &= \frac{x_p-x_1}{x_0-x_1}x_0+\frac{x_p-x_0}{x_1-x_0}x_1
    = \frac{x_0x_p-x_0x_1+x_1x_0-x_1x_p}{x_0-x_1}
    = x_p\frac{x_0-x_1}{x_0-x_1} = x_p
  \end{split}
\end{equation}

\begin{equation}
  \begin{split}
    \sum_{i=0}^{3}a_iy_i &= \frac{x_p-x_1}{x_0-x_1}\frac{y_p-y_2}{y_0-y_2}y_0+
    \frac{x_p-x_0}{x_1-x_0}\frac{y_p-y_2}{y_0-y_2}y_0 +
    \frac{x_p-x_1}{x_0-x_1}\frac{y_p-y_0}{y_2-y_0}y_2 +
    \frac{x_p-x_0}{x_1-x_0}\frac{y_p-y_0}{y_2-y_0}y_2 \\
    &= \frac{y_p-y_2}{y_0-y_2}y_0+\frac{y_p-y_0}{y_2-y_0}y_2
    = \frac{y_0y_p-y_0y_2+y_2y_0-y_2y_p}{y_0-y_2}
    = y_p\frac{y_0-y_2}{y_0-y_2} = y_p
  \end{split}
\end{equation}

\subsubsection{Contour Extraction}

Contour Extraction works clockwise around points ``inside'' (i.e. whose value is greater than the contour line value). There are 5 cases (and rotational symmetric variants) to consider.

\begin{figure}[!htbp]
  \begin{center}
    \jnxfig[angle=0,scale=1]{contour110.eps}{contour110.pdf}
    \caption{Contour 1--1}
    \label{fig:contour110}
  \end{center}
\end{figure}

\begin{figure}[!htbp]
  \begin{center}
    \jnxfig[angle=0,scale=1]{contour111.eps}{contour111.pdf}
    \caption{Contour 1--1}
    \label{fig:contour111}
  \end{center}
\end{figure}

\begin{figure}[!htbp]
  \begin{center}
    \jnxfig[angle=0,scale=1]{contour1.eps}{contour1.pdf}
    \caption{Contour 1}
    \label{fig:contour1}
  \end{center}
\end{figure}

\begin{figure}[!htbp]
  \begin{center}
    \jnxfig[angle=0,scale=1]{contour2.eps}{contour2.pdf}
    \caption{Contour 2}
    \label{fig:contour2}
  \end{center}
\end{figure}

\begin{figure}[!htbp]
  \begin{center}
    \jnxfig[angle=0,scale=1]{contour3.eps}{contour3.pdf}
    \caption{Contour 3}
    \label{fig:contour3}
  \end{center}
\end{figure}
\clearpage

\ifphymodel
\subsection{Physical Airframe / Engine Model}

\subsubsection{Propeller}

\begin{figure}[!htbp]
  \begin{center}
    \jnxfig[angle=0,width=\linewidth]{propafcpct.eps}{propafcpct.pdf}
    \caption{$P_{CP}$/$P_{CT}$ versus Propeller Activity Factor $AF$}
    \label{fig:propafcpct}
  \end{center}
\end{figure}

\begin{figure}[!htbp]
  \begin{center}
    \jnxfig[angle=0,width=\linewidth]{propafcpcterr.eps}{propafcpcterr.pdf}
    \caption{$P_{CP}$/$P_{CT}$ versus Propeller Activity Factor $AF$ Approximation Error}
    \label{fig:propafcpcterr}
  \end{center}
\end{figure}

\begin{figure}[!htbp]
  \begin{center}
    \jnxfig[angle=0,width=\linewidth]{propcpstal.eps}{propcpstal.pdf}
    \caption{$C_P$ where 50\% of the propeller is stalled versus $J$}
    \label{fig:propcpstal}
  \end{center}
\end{figure}

\begin{figure}[!htbp]
  \begin{center}
    \jnxfig[angle=0,width=\linewidth]{propcpstalerr.eps}{propcpstalerr.pdf}
    \caption{$C_P$ where 50\% of the propeller is stalled versus $J$ Approximation Error}
    \label{fig:propcpstalerr}
  \end{center}
\end{figure}

\begin{figure}[!htbp]
  \begin{center}
    \jnxfig[angle=0,width=\linewidth]{propctstal.eps}{propctstal.pdf}
    \caption{$C_T$ where 50\% of the propeller is stalled versus $J$}
    \label{fig:propctstal}
  \end{center}
\end{figure}

\begin{figure}[!htbp]
  \begin{center}
    \jnxfig[angle=0,width=\linewidth]{propctstalerr.eps}{propctstalerr.pdf}
    \caption{$C_T$ where 50\% of the propeller is stalled versus $J$ Approximation Error}
    \label{fig:propctstalerr}
  \end{center}
\end{figure}

\begin{figure}[!htbp]
  \begin{center}
    \jnxfig[angle=0,width=\linewidth]{proppbl.eps}{proppbl.pdf}
    \caption{$P_{BL}$}
    \label{fig:proppbl}
  \end{center}
\end{figure}
\clearpage

\begin{figure}[!htbp]
  \begin{center}
    \jnxfig[angle=0,width=\linewidth]{propbldangcp1.eps}{propbldangcp1.pdf}
    \caption{$C_P$ versus Blade Angle $\beta_{3/4}$ for a 2 Blade Propeller}
    \label{fig:propbldangcp1}
  \end{center}
\end{figure}

\begin{figure}[!htbp]
  \begin{center}
    \jnxfig[angle=0,width=\linewidth]{propbldangct1.eps}{propbldangct1.pdf}
    \caption{$C_T$ versus Blade Angle $\beta_{3/4}$ for a 2 Blade Propeller}
    \label{fig:propbldangct1}
  \end{center}
\end{figure}

\begin{figure}[!htbp]
  \begin{center}
    \jnxfig[angle=0,width=\linewidth]{propbldangcp1d.eps}{propbldangcp1d.pdf}
    \caption{$C_P$ versus Blade Angle $\beta_{3/4}$ for a 2 Blade Propeller}
    \label{fig:propbldangcp1d}
  \end{center}
\end{figure}

\begin{figure}[!htbp]
  \begin{center}
    \jnxfig[angle=0,width=\linewidth]{propbldangct1d.eps}{propbldangct1d.pdf}
    \caption{$C_T$ versus Blade Angle $\beta_{3/4}$ for a 2 Blade Propeller}
    \label{fig:propbldangct1d}
  \end{center}
\end{figure}

\begin{figure}[!htbp]
  \begin{center}
    \jnxfig[angle=0,width=\linewidth]{propbldangcp2.eps}{propbldangcp2.pdf}
    \caption{$C_P$ versus Blade Angle $\beta_{3/4}$ for a 4 Blade Propeller}
    \label{fig:propbldangcp2}
  \end{center}
\end{figure}

\begin{figure}[!htbp]
  \begin{center}
    \jnxfig[angle=0,width=\linewidth]{propbldangct2.eps}{propbldangct2.pdf}
    \caption{$C_T$ versus Blade Angle $\beta_{3/4}$ for a 4 Blade Propeller}
    \label{fig:propbldangct2}
  \end{center}
\end{figure}

\begin{figure}[!htbp]
  \begin{center}
    \jnxfig[angle=0,width=\linewidth]{propbldangcp2d.eps}{propbldangcp2d.pdf}
    \caption{$C_P$ versus Blade Angle $\beta_{3/4}$ for a 4 Blade Propeller}
    \label{fig:propbldangcp2d}
  \end{center}
\end{figure}

\begin{figure}[!htbp]
  \begin{center}
    \jnxfig[angle=0,width=\linewidth]{propbldangct2d.eps}{propbldangct2d.pdf}
    \caption{$C_T$ versus Blade Angle $\beta_{3/4}$ for a 4 Blade Propeller}
    \label{fig:propbldangct2d}
  \end{center}
\end{figure}

\begin{figure}[!htbp]
  \begin{center}
    \jnxfig[angle=0,width=\linewidth]{propbldangcp3.eps}{propbldangcp3.pdf}
    \caption{$C_P$ versus Blade Angle $\beta_{3/4}$ for a 6 Blade Propeller}
    \label{fig:propbldangcp3}
  \end{center}
\end{figure}

\begin{figure}[!htbp]
  \begin{center}
    \jnxfig[angle=0,width=\linewidth]{propbldangct3.eps}{propbldangct3.pdf}
    \caption{$C_T$ versus Blade Angle $\beta_{3/4}$ for a 6 Blade Propeller}
    \label{fig:propbldangct3}
  \end{center}
\end{figure}

\begin{figure}[!htbp]
  \begin{center}
    \jnxfig[angle=0,width=\linewidth]{propbldangcp3d.eps}{propbldangcp3d.pdf}
    \caption{$C_P$ versus Blade Angle $\beta_{3/4}$ for a 6 Blade Propeller}
    \label{fig:propbldangcp3d}
  \end{center}
\end{figure}

\begin{figure}[!htbp]
  \begin{center}
    \jnxfig[angle=0,width=\linewidth]{propbldangct3d.eps}{propbldangct3d.pdf}
    \caption{$C_T$ versus Blade Angle $\beta_{3/4}$ for a 6 Blade Propeller}
    \label{fig:propbldangct3d}
  \end{center}
\end{figure}

\begin{figure}[!htbp]
  \begin{center}
    \jnxfig[angle=0,width=\linewidth]{propbldangcp4.eps}{propbldangcp4.pdf}
    \caption{$C_P$ versus Blade Angle $\beta_{3/4}$ for a 8 Blade Propeller}
    \label{fig:propbldangcp4}
  \end{center}
\end{figure}

\begin{figure}[!htbp]
  \begin{center}
    \jnxfig[angle=0,width=\linewidth]{propbldangct4.eps}{propbldangct4.pdf}
    \caption{$C_T$ versus Blade Angle $\beta_{3/4}$ for a 8 Blade Propeller}
    \label{fig:propbldangct4}
  \end{center}
\end{figure}

\begin{figure}[!htbp]
  \begin{center}
    \jnxfig[angle=0,width=\linewidth]{propbldangcp4d.eps}{propbldangcp4d.pdf}
    \caption{$C_P$ versus Blade Angle $\beta_{3/4}$ for a 8 Blade Propeller}
    \label{fig:propbldangcp4d}
  \end{center}
\end{figure}

\begin{figure}[!htbp]
  \begin{center}
    \jnxfig[angle=0,width=\linewidth]{propbldangct4d.eps}{propbldangct4d.pdf}
    \caption{$C_T$ versus Blade Angle $\beta_{3/4}$ for a 8 Blade Propeller}
    \label{fig:propbldangct4d}
  \end{center}
\end{figure}
\clearpage

\begin{figure}[!htbp]
  \begin{center}
    \jnxfig[angle=0,width=\linewidth]{hartzellf7666a2af.eps}{hartzellf7666a2af.pdf}
    \caption{Hartzell F7666A-2 Activity Factor}
    \label{fig:hartzellf7666a2af}
  \end{center}
\end{figure}

\nocite{prop:nasacr114289,prop:nasacr2066}

\begin{enumerate}
\item $\rho_0/\rho$ Density Ratio
\item $N_0$ Rated Engine RPM (RPM at which SHP is given) (in $\mathrm{min}^{-1}$)
\item $N$ Engine RPM (in $\mathrm{min}^{-1}$)
\item $D$ Propeller Diameter (in $\mathrm{ft}$)
\item $C_P=\frac{\frac{N}{N_0}\cdot\mathrm{SHP}\cdot\rho_0/\rho\cdot{}10^{11}}{2\cdot{}N^3\cdot{}D^5}$
\item $J=\frac{101.4\cdot{}V_K}{N\cdot{}D}$ ($V_K$ free stream velocity in $\mathrm{kts}$)
\item $P_{AF}=f_{PAF}(AF)$ activity factor correction (Figure \ref{fig:propafcpct})
\item $C_{P_E}=C_P\cdot{}P_{AF}$
\item $\beta_{3/4}=f^{-1}_{\beta_{3/4}\rightarrow{}C_{P_E}}(J,C_{P_E})$ ($\beta_{3/4}$ is the blade angle)
  (Figures \ref{fig:propbldangcp1}, \ref{fig:propbldangcp2}, \ref{fig:propbldangcp3}, \ref{fig:propbldangcp4})
\item $C_{T_E}=f_{\beta_{3/4}\rightarrow{}C_{T_E}}(J,\beta_{3/4})$
  (Figures \ref{fig:propbldangct1}, \ref{fig:propbldangct2}, \ref{fig:propbldangct3}, \ref{fig:propbldangct4})
\item $T_{AF}=f_{TAF}(AF)$ activity factor correction (Figure \ref{fig:propafcpct})
\item $C_T=\frac{C_{T_E}}{T_{AF}}$
\item $T=\frac{0.661\cdot{}10^{-6}\cdot{}C_T\cdot{}N^2\cdot{}D^4}{\rho_0/\rho}$ (Thrust in $\mathrm{lbf}$)
\item $\eta=\frac{C_T}{C_P}\cdot{}J$ Propeller efficiency
\end{enumerate}

\paragraph{Fixed Pitch Propeller}

\begin{equation}
  \epsilon=\frac{\mathrm{SHP}\cdot\rho_0/\rho\cdot{}10^{11}}{2\cdot{}N_0\cdot{}N^2\cdot{}D^5}\cdot{}P_{AF}
  -f_{\beta_{3/4}\rightarrow{}C_{P_E}}\left(\frac{101.4\cdot{}V_K}{N\cdot{}D},\beta_{3/4}\right)
\end{equation}

Solve $N$ for $\epsilon=0$.

\begin{equation}
  \frac{\partial}{\partial{}N}\epsilon=-\frac{\mathrm{SHP}\cdot\rho_0/\rho\cdot{}10^{11}}{N_0\cdot{}N^3\cdot{}D^5}\cdot{}P_{AF}
  +\frac{\partial}{\partial{}C_{P_E}}f_{\beta_{3/4}\rightarrow{}C_{P_E}}\left(\frac{101.4\cdot{}V_K}{N\cdot{}D},\beta_{3/4}\right)\cdot{}\frac{101.4\cdot{}V_K}{N^2\cdot{}D}
\end{equation}

\subsubsection{Bootstrap Method}

See {\tt http://www.allstar.fiu.edu/aero/airper-ba.htm}.

Bootstrap data plate for Piper Cherokee Arrow 200.

\noindent\begin{tabular}{lrll}
  Bootstrap Data Plate Item & Value & Units & Aircraft Subsystem \\
  Wing area, $S$ & 160 & $\mathrm{ft}^2$ & Airframe \\
  Wing aspect ratio, $A$ & 5.625 & & Airframe \\
  Rated MSL torque, $M_0$ & 389.05 & $\mathrm{ft}\cdot\mathrm{lbf}$ & Engine \\
  Altitude drop-off parameter, $C$ & 0.12 & & Engine \\
  Propeller diameter, $d$ & 6.333 & $\mathrm{ft}$ & Propeller \\
  Parasite drag coefficient, $C_{D_0}$ & 0.030662 & & Airframe \\
  Airplane efficiency factor, $e$ & 0.60648 & & Airframe \\
  Propeller polar slope, $m$ & 2.073704 & & Propeller \\
  Propeller polar intercept, $b$ & -0.054411 & & Propeller \\
\end{tabular} \\

Wing aspect ratio: $A = B^2/S$, where B is the wing span. \\

Mean sea level (MSL) full--throttle rated torque $M_0 = \frac{P_0}{2\pi{}n_0}$,
where $P_0$ is the rated power, $n_0$ is the rated propeller revolutions per second. \\

For the Arrow 200, $P_0 = 200\mathrm{HP} = 110000\frac{\mathrm{ft}\cdot\mathrm{lbf}}{\mathrm{s}}$,
and $n_0 = \frac{\mathrm{RPM}}{60\mathrm{s}} = \frac{2700}{60\mathrm{s}} = 45\mathrm{s}^{-1}$. \\

The proportional mechanical power loss independent of altitude, $C$, can almost always be 
taken as 0.12. \\

$M(\sigma) = \Phi(\sigma)\cdot{}M_0$. $\Phi(\sigma) = \frac{\sigma - C}{1 - C}$. $\sigma$ is the
relative density. \\

Parasite drag and efficiency. $V_T\cdot\sqrt{\sigma}=V_C$. $\gamma_{bg}=\tan^{-1}(\frac{1}{r_{bg}})$.
$r_{bg}$ is the glide ratio, $\gamma_{bg}$ the glide angle, $W$ the aircraft weight.
$\rho_0=0.002377\frac{\mathrm{slug}}{\mathrm{ft}^3}$.
$C_{D_0}=\frac{W\cdot{}\sin(\gamma_{bg})}{\rho_0\cdot{}V_{Cbg}^2\cdot{}S}$.
$e=\frac{4\cdot{}C_{D_0}}{\pi\cdot{}A\cdot\tan^2(\gamma_{bg})}$. \\

For the Arrow 200, $W=2600\mathrm{lbf}$, $r_{bg}=\frac{20\mathrm{nmi}}{13000\mathrm{ft}}=9.3479$,
$\gamma_{bg}=6.1061$, $V_{Cbg}=105\mathrm{MPH}=154\frac{\mathrm{ft}}{\mathrm{s}}$,
$C_{D_0}=0.030662$, $e=0.60648$. \\

Propeller polars. $V_{Tx}$ true speed for best glide angle.
$b=\frac{S\cdot{}C_{D_0}}{2\cdot{}d^2}-\frac{2\cdot{}W^2}{\rho^2\cdot{}d^2\cdot{}S\cdot\pi\cdot{}e\cdot{}A\cdot{}V_{Cx}^4}$.
$m=\frac{2\cdot{}n_0\cdot{}d\cdot{}W^2}{\Phi(\sigma)\cdot{}P_0\cdot\rho\cdot{}S\cdot\pi\cdot{}e\cdot{}A}\cdot\left(\frac{1}{V_{CM}^2}-\frac{V_{CM}^2}{V_{Cx}^4}\right)$. \\

For the Arrow 200, $V_{Cx}=80\mathrm{MPH}=117.33333\frac{\mathrm{ft}}{\mathrm{s}}$,
$V_{CM}=174\mathrm{MPH}=255.2\frac{\mathrm{ft}}{\mathrm{s}}$ @ 5000ft,
$V_{CM}=168\mathrm{MPH}=246.4\frac{\mathrm{ft}}{\mathrm{s}}$ @ 10000ft, $\sigma=0.73875$,
$\Phi(\sigma)=0.70312$, $b=-0.12241$, $m=5.5720$. \\


\noindent\begin{tabular}{rr}
Alt & Density $\rho$ \\
$\mathrm{ft}$ & $10^{-4}\frac{\mathrm{slugs}}{\mathrm{ft}^3}$ \\
-5000 &	27.45 \\
0     &	23.77 \\
5000  &	20.48 \\
10000 &	17.56 \\
15000 &	14.96 \\
\end{tabular} \\

\paragraph{Thrust / Power}

Propeller Advance $J=\frac{V_T}{n\cdot{}d}$, $V_T$ (true) speed, $n$ revolutions per second, $d$ propeller diameter.\\

Propeller Thrust $T=C_T\cdot\rho\cdot{}n^2\cdot{}d^4$.\\

Propeller Power $P=C_P\cdot\rho\cdot{}n^3\cdot{}d^5$.\\

Propeller Efficiency $\eta=J\cdot\frac{C_T}{C_P}=\frac{V_T\cdot{}T}{P}$.\\

Propeller Polar $\frac{C_T}{J^2}=m\cdot\frac{C_P}{J^2}+b$.\\

$C_P=\frac{P}{\rho\cdot{}n^3\cdot{}d^5}.$\\

$C_T=m\cdot{}C_P+J^2\cdot{}b=m\cdot\frac{P}{\rho\cdot{}n^3\cdot{}d^5}+J^2\cdot{}b$.\\

$T=\frac{m\cdot{}P}{n\cdot{}d}+J^2\cdot{}b\cdot\rho\cdot{}n^2\cdot{}d^4
=\frac{m\cdot{}P}{n\cdot{}d}+V_T^2\cdot{}b\cdot\rho\cdot{}d^2
=\frac{m\cdot{}P}{n\cdot{}d}+V_C^2\cdot{}b\cdot\rho_0\cdot{}d^2$.\\

$P=\frac{n\cdot{}d}{m}\cdot\left(T-V_C^2\cdot{}b\cdot\rho_0\cdot{}d^2\right)$.\\


\paragraph{Drag / Lift}

$C_D = C_{D_0} + \frac{C_L^2}{\pi\cdot{}A\cdot{}e}$.\\

$L=\frac{1}{2}\cdot{}C_L\cdot\rho\cdot{}V_T^2\cdot{}S$. \\

$D=\frac{1}{2}\cdot{}C_D\cdot\rho\cdot{}V_T^2\cdot{}S$. \\

$C_L=\frac{W}{\frac{1}{2}\cdot\rho\cdot{}V_T^2\cdot{}S}$

\begin{equation}
  \begin{split}
    D&=\frac{1}{2}\cdot{}C_{D_0}\cdot\rho\cdot{}V_T^2\cdot{}S
    + \frac{1}{2}\cdot{}\frac{W^2}{(\frac{1}{2}\cdot\rho\cdot{}V_T^2\cdot{}S)^2\cdot\pi\cdot{}A\cdot{}e}\cdot\rho\cdot{}V_T^2\cdot{}S \\
    &=\frac{1}{2}\cdot{}C_{D_0}\cdot\rho\cdot{}V_T^2\cdot{}S
    + \frac{W^2}{\frac{1}{2}\cdot\rho\cdot{}V_T^2\cdot{}S\cdot\pi\cdot{}A\cdot{}e} \\
    &=\frac{1}{2}\cdot{}C_{D_0}\cdot\rho_0\cdot{}V_C^2\cdot{}S
    + \frac{W^2}{\frac{1}{2}\cdot\rho_0\cdot{}V_C^2\cdot{}S\cdot\pi\cdot{}A\cdot{}e}
  \end{split}
\end{equation}

\begin{equation}
  0 = \frac{1}{2}\cdot{}C_{D_0}\cdot\rho_0^2\cdot{}V_C^4\cdot{}S
    - D\cdot\rho_0\cdot{}V_C^2
    + \frac{W^2}{\frac{1}{2}\cdot{}S\cdot\pi\cdot{}A\cdot{}e}
\end{equation}

\begin{equation}
  \rho_0\cdot{}V_C^2 = \rho\cdot{}V_T^2 = \frac{D}{C_{D_0}\cdot{}S}\pm\sqrt{\frac{D^2}{C_{D_0}^2\cdot{}S^2}-\frac{4\cdot{}W^2}{C_{D_0}\cdot{}S^2\cdot\pi\cdot{}A\cdot{}e}}
\end{equation}

\paragraph{Unaccelerated Level Flight}

\begin{equation}
  \rho\cdot{}V_{T,max}^2 = \frac{\frac{m\cdot{}P}{n\cdot{}d}+\rho\cdot{}V_{T,max}^2\cdot{}b\cdot{}d^2}{C_{D_0}\cdot{}S}
  +\sqrt{\frac{\left(\frac{m\cdot{}P}{n\cdot{}d}+\rho\cdot{}V_{T,max}^2\cdot{}b\cdot{}d^2\right)^2}{C_{D_0}^2\cdot{}S^2}-\frac{4\cdot{}W^2}{C_{D_0}\cdot{}S^2\cdot\pi\cdot{}A\cdot{}e}}
\end{equation}

\begin{equation}
  \rho\cdot{}V_{T,min}^2 = \frac{\frac{m\cdot{}P}{n\cdot{}d}+\rho\cdot{}V_{T,min}^2\cdot{}b\cdot{}d^2}{C_{D_0}\cdot{}S}
  -\sqrt{\frac{\left(\frac{m\cdot{}P}{n\cdot{}d}+\rho\cdot{}V_{T,min}^2\cdot{}b\cdot{}d^2\right)^2}{C_{D_0}^2\cdot{}S^2}-\frac{4\cdot{}W^2}{C_{D_0}\cdot{}S^2\cdot\pi\cdot{}A\cdot{}e}}
\end{equation}

Initial Guess:

\begin{equation}
  \rho\cdot{}\hat{V}_{T,max}^2 = \frac{m\cdot{}P}{n\cdot{}d\cdot{}C_{D_0}\cdot{}S}
  +\sqrt{\frac{m^2\cdot{}P^2}{n^2\cdot{}d^2\cdot{}C_{D_0}^2\cdot{}S^2}-\frac{4\cdot{}W^2}{C_{D_0}\cdot{}S^2\cdot\pi\cdot{}A\cdot{}e}}
\end{equation}

\begin{equation}
  \rho\cdot{}\hat{V}_{T,min}^2 = \frac{m\cdot{}P}{n\cdot{}d\cdot{}C_{D_0}\cdot{}S}
  -\sqrt{\frac{m^2\cdot{}P^2}{n^2\cdot{}d^2\cdot{}C_{D_0}^2\cdot{}S^2}-\frac{4\cdot{}W^2}{C_{D_0}\cdot{}S^2\cdot\pi\cdot{}A\cdot{}e}}
\end{equation}

Error:

\begin{equation}
  \epsilon = \frac{\frac{m\cdot{}P}{n\cdot{}d}+\rho\cdot{}\hat{V}_{T,max}^2\cdot{}b\cdot{}d^2}{C_{D_0}\cdot{}S} - \rho\cdot{}\hat{V}_{T,max}^2
  +\sqrt{\frac{\left(\frac{m\cdot{}P}{n\cdot{}d}+\rho\cdot{}\hat{V}_{T,max}^2\cdot{}b\cdot{}d^2\right)^2}{C_{D_0}^2\cdot{}S^2}-\frac{4\cdot{}W^2}{C_{D_0}\cdot{}S^2\cdot\pi\cdot{}A\cdot{}e}}
\end{equation}

\begin{equation}
  \epsilon = \frac{\frac{m\cdot{}P}{n\cdot{}d}+\rho\cdot{}\hat{V}_{T,max}^2\cdot{}b\cdot{}d^2}{C_{D_0}\cdot{}S} - \rho\cdot{}\hat{V}_{T,max}^2
  -\sqrt{\frac{\left(\frac{m\cdot{}P}{n\cdot{}d}+\rho\cdot{}\hat{V}_{T,max}^2\cdot{}b\cdot{}d^2\right)^2}{C_{D_0}^2\cdot{}S^2}-\frac{4\cdot{}W^2}{C_{D_0}\cdot{}S^2\cdot\pi\cdot{}A\cdot{}e}}
\end{equation}

\begin{equation}
  \frac{\partial}{\partial\rho\cdot{}\hat{V}_{T,max}^2}\epsilon = \frac{b\cdot{}d^2}{C_{D_0}\cdot{}S}-1+
  \frac{b\cdot{}d^2\cdot\left(\frac{m\cdot{}P}{n\cdot{}d}+\rho\cdot{}\hat{V}_{T,max}^2\cdot{}b\cdot{}d^2\right)}
  {C_{D_0}\cdot{}S\cdot\sqrt{\left(\frac{m\cdot{}P}{n\cdot{}d}+\rho\cdot{}\hat{V}_{T,max}^2\cdot{}b\cdot{}d^2\right)^2-
      \frac{4\cdot{}W^2\cdot{}C_{D_0}}{\pi\cdot{}A\cdot{}e}}}
\end{equation}

\begin{equation}
  \frac{\partial}{\partial\rho\cdot{}\hat{V}_{T,min}^2}\epsilon = \frac{b\cdot{}d^2}{C_{D_0}\cdot{}S}-1-
  \frac{b\cdot{}d^2\cdot\left(\frac{m\cdot{}P}{n\cdot{}d}+\rho\cdot{}\hat{V}_{T,min}^2\cdot{}b\cdot{}d^2\right)}
  {C_{D_0}\cdot{}S\cdot\sqrt{\left(\frac{m\cdot{}P}{n\cdot{}d}+\rho\cdot{}\hat{V}_{T,min}^2\cdot{}b\cdot{}d^2\right)^2-
      \frac{4\cdot{}W^2\cdot{}C_{D_0}}{\pi\cdot{}A\cdot{}e}}}
\end{equation}

\paragraph{Rate of Climb}

\begin{equation}
  \frac{m\cdot{}P}{n\cdot{}d}+V_C^2\cdot{}b\cdot\rho_0\cdot{}d^2=
  \frac{1}{2}\cdot{}C_{D_0}\cdot\rho_0\cdot{}V_C^2\cdot{}S
    + \frac{W^2\cdot\cos^2(\alpha)}{\frac{1}{2}\cdot\rho_0\cdot{}V_C^2\cdot{}S\cdot\pi\cdot{}A\cdot{}e}+W\sin(\alpha)
\end{equation}

\begin{equation}
  \frac{m\cdot{}P}{n\cdot{}d}+V_C^2\cdot\rho_0\cdot\left(b\cdot{}d^2-\frac{1}{2}\cdot{}C_{D_0}\cdot{}S\right)=
  \frac{W^2\cdot\cos^2(\alpha)}{\frac{1}{2}\cdot\rho_0\cdot{}V_C^2\cdot{}S\cdot\pi\cdot{}A\cdot{}e}+W\sin(\alpha)
\end{equation}

\begin{equation}
  \frac{m\cdot{}P}{n\cdot{}d}+V_C^2\cdot\rho_0\cdot\left(b\cdot{}d^2-\frac{1}{2}\cdot{}C_{D_0}\cdot{}S\right)
  -\frac{W^2}{\frac{1}{2}\cdot\rho_0\cdot{}V_C^2\cdot{}S\cdot\pi\cdot{}A\cdot{}e}=
  W\sin(\alpha)-\frac{W^2\cdot\sin^2(\alpha)}{\frac{1}{2}\cdot\rho_0\cdot{}V_C^2\cdot{}S\cdot\pi\cdot{}A\cdot{}e}
\end{equation}

\begin{equation}
  \begin{split}
    \sin(\alpha)=&\frac{\rho_0\cdot{}V_C^2\cdot{}S\cdot\pi\cdot{}A\cdot{}e}{4\cdot{}W^2}\cdot \\
    &\left(W\pm\sqrt{W^2-\frac{8\cdot{}W^2}{\rho_0\cdot{}V_C^2\cdot{}S\cdot\pi\cdot{}A\cdot{}e}\cdot
        \left(\frac{m\cdot{}P}{n\cdot{}d}+V_C^2\cdot\rho_0\cdot\left(b\cdot{}d^2-\frac{1}{2}\cdot{}C_{D_0}\cdot{}S\right)
          -\frac{W^2}{\frac{1}{2}\cdot\rho_0\cdot{}V_C^2\cdot{}S\cdot\pi\cdot{}A\cdot{}e}\right)}\right)
  \end{split}
\end{equation}

\begin{equation}
  \sin(\alpha)\approx\frac{1}{W}\cdot\left(\frac{m\cdot{}P}{n\cdot{}d}+V_C^2\cdot\rho_0\cdot\left(b\cdot{}d^2-\frac{1}{2}\cdot{}C_{D_0}\cdot{}S\right)
  -\frac{W^2}{\frac{1}{2}\cdot\rho_0\cdot{}V_C^2\cdot{}S\cdot\pi\cdot{}A\cdot{}e}\right)
\end{equation}

\begin{equation}
  \begin{split}
    \mathrm{ROC} &= V_T\cdot\sin(\alpha) = V_C\cdot\sqrt{\frac{\rho_0}{\rho}}\cdot\sin(\alpha) \\
    &\approx{}V_C\cdot\sqrt{\frac{\rho_0}{\rho}}\cdot\frac{1}{W}\cdot\left(\frac{m\cdot{}P}{n\cdot{}d}+V_C^2\cdot\rho_0\cdot\left(b\cdot{}d^2-\frac{1}{2}\cdot{}C_{D_0}\cdot{}S\right)
      -\frac{W^2}{\frac{1}{2}\cdot\rho_0\cdot{}V_C^2\cdot{}S\cdot\pi\cdot{}A\cdot{}e}\right)
  \end{split}
\end{equation}

\paragraph{Rate of Climb 2}

\begin{equation}
  \begin{split}
    \mathrm{ROC} &= \frac{V_T}{W}(T-D)
  \end{split}
\end{equation}

\begin{figure}[!htbp]
  \begin{center}
    \jnxfig[angle=0,width=\linewidth]{p28ratmo.eps}{p28ratmo.pdf}
    \caption{ICAO Standard Atmosphere}
    \label{fig:p28ratmo}
  \end{center}
\end{figure}

\begin{figure}[!htbp]
  \begin{center}
    \jnxfig[angle=0,width=\linewidth]{p28rio360pwr.eps}{p28rio360pwr.pdf}
    \caption{IO-360-C Engine Performance versus Altitude}
    \label{fig:p28rio360pwr}
  \end{center}
\end{figure}

\begin{figure}[!htbp]
  \begin{center}
    \jnxfig[angle=0,width=\linewidth]{p28rio360pwrdensity.eps}{p28rio360pwrdensity.pdf}
    \caption{IO-360-C Engine Performance versus Altitude}
    \label{fig:p28rio360pwrdensity}
  \end{center}
\end{figure}
\clearpage

\begin{figure}[!htbp]
  \begin{center}
    \jnxfig[angle=0,width=\linewidth]{p28rslgndrun.eps}{p28rslgndrun.pdf}
    \caption{P28R-200B Sea Level Ground Run}
    \label{fig:p28rslgndrun}
  \end{center}
\end{figure}

\begin{figure}[!htbp]
  \begin{center}
    \jnxfig[angle=0,width=\linewidth]{p28rslgndrunprop.eps}{p28rslgndrunprop.pdf}
    \caption{P28R-200B Sea Level Ground Run}
    \label{fig:p28rslgndrunprop}
  \end{center}
\end{figure}

\begin{figure}[!htbp]
  \begin{center}
    \jnxfig[angle=0,width=\linewidth]{p28rclimbforce.eps}{p28rclimbforce.pdf}
    \caption{P28R-200B Climb Performance}
    \label{fig:p28rclimbforce}
  \end{center}
\end{figure}

\begin{figure}[!htbp]
  \begin{center}
    \jnxfig[angle=0,width=\linewidth]{p28rclimbprop.eps}{p28rclimbprop.pdf}
    \caption{P28R-200B Climb Propeller Efficiency / Pitch}
    \label{fig:p28rclimbprop}
  \end{center}
\end{figure}

\begin{figure}[!htbp]
  \begin{center}
    \jnxfig[angle=0,width=\linewidth]{p28rclimbforcegd.eps}{p28rclimbforcegd.pdf}
    \caption{P28R-200B Climb Performance}
    \label{fig:p28rclimbforcegd}
  \end{center}
\end{figure}

\begin{figure}[!htbp]
  \begin{center}
    \jnxfig[angle=0,width=\linewidth]{p28rclimbpropgd.eps}{p28rclimbpropgd.pdf}
    \caption{P28R-200B Climb Propeller Efficiency / Pitch}
    \label{fig:p28rclimbpropgd}
  \end{center}
\end{figure}
\clearpage

\begin{figure}[!htbp]
  \begin{center}
    \jnxfig[angle=0,width=\linewidth]{p28rroc2700.eps}{p28rroc2700.pdf}
    \caption{P28R-200B Rate of Climb, 2700 RPM}
    \label{fig:p28rroc2700}
  \end{center}
\end{figure}

\begin{figure}[!htbp]
  \begin{center}
    \jnxfig[angle=0,width=\linewidth]{p28rroceff2700.eps}{p28rroceff2700.pdf}
    \caption{P28R-200B Climb Propeller Efficiency, 2700 RPM}
    \label{fig:p28rroceff2700}
  \end{center}
\end{figure}

\begin{figure}[!htbp]
  \begin{center}
    \jnxfig[angle=0,width=\linewidth]{p28rrocpitch2700.eps}{p28rrocpitch2700.pdf}
    \caption{P28R-200B Climb Propeller Pitch, 2700 RPM}
    \label{fig:p28rrocpitch2700}
  \end{center}
\end{figure}

\begin{figure}[!htbp]
  \begin{center}
    \jnxfig[angle=0,width=\linewidth]{p28rroc2500.eps}{p28rroc2500.pdf}
    \caption{P28R-200B Rate of Climb, 2500 RPM}
    \label{fig:p28rroc2500}
  \end{center}
\end{figure}

\begin{figure}[!htbp]
  \begin{center}
    \jnxfig[angle=0,width=\linewidth]{p28rroceff2500.eps}{p28rroceff2500.pdf}
    \caption{P28R-200B Climb Propeller Efficiency, 2500 RPM}
    \label{fig:p28rroceff2500}
  \end{center}
\end{figure}

\begin{figure}[!htbp]
  \begin{center}
    \jnxfig[angle=0,width=\linewidth]{p28rrocpitch2500.eps}{p28rrocpitch2500.pdf}
    \caption{P28R-200B Climb Propeller Pitch, 2500 RPM}
    \label{fig:p28rrocpitch2500}
  \end{center}
\end{figure}
\clearpage

\begin{figure}[!htbp]
  \begin{center}
    \jnxfig[angle=0,width=\linewidth]{p28rcruisespeed.eps}{p28rcruisespeed.pdf}
    \caption{P28R-200B Cruise Speed}
    \label{fig:p28rcruisespeed}
  \end{center}
\end{figure}

\begin{figure}[!htbp]
  \begin{center}
    \jnxfig[angle=0,width=\linewidth]{p28rcruiseeff.eps}{p28rcruiseeff.pdf}
    \caption{P28R-200B Cruise Propeller Efficiency}
    \label{fig:p28rcruiseeff}
  \end{center}
\end{figure}

\begin{figure}[!htbp]
  \begin{center}
    \jnxfig[angle=0,width=\linewidth]{p28rcruisepitch.eps}{p28rcruisepitch.pdf}
    \caption{P28R-200B Cruise Propeller Pitch}
    \label{fig:p28rcruisepitch}
  \end{center}
\end{figure}

\begin{figure}[!htbp]
  \begin{center}
    \jnxfig[angle=0,width=\linewidth]{p28rcruisespeedgd.eps}{p28rcruisespeedgd.pdf}
    \caption{P28R-200B Cruise Speed, Gear Down}
    \label{fig:p28rcruisespeedgd}
  \end{center}
\end{figure}

\begin{figure}[!htbp]
  \begin{center}
    \jnxfig[angle=0,width=\linewidth]{p28rcruiseeffgd.eps}{p28rcruiseeffgd.pdf}
    \caption{P28R-200B Cruise Propeller Efficiency, Gear Down}
    \label{fig:p28rcruiseeffgd}
  \end{center}
\end{figure}

\begin{figure}[!htbp]
  \begin{center}
    \jnxfig[angle=0,width=\linewidth]{p28rcruisepitchgd.eps}{p28rcruisepitchgd.pdf}
    \caption{P28R-200B Cruise Propeller Pitch, Gear Down}
    \label{fig:p28rcruisepitchgd}
  \end{center}
\end{figure}
\clearpage

\begin{figure}[!htbp]
  \begin{center}
    \jnxfig[angle=0,width=\linewidth]{p28rbestglidepp14.eps}{p28rbestglidepp14.pdf}
    \caption{P28R-200B Best Glide, Minimum Propeller Pitch}
    \label{fig:p28rbestglidepp14}
  \end{center}
\end{figure}

\begin{figure}[!htbp]
  \begin{center}
    \jnxfig[angle=0,width=\linewidth]{p28rbestglideproppp14.eps}{p28rbestglideproppp14.pdf}
    \caption{P28R-200B Best Glide Propeller Parameters, Minimum Propeller Pitch}
    \label{fig:p28rbestglideproppp14}
  \end{center}
\end{figure}

\begin{figure}[!htbp]
  \begin{center}
    \jnxfig[angle=0,width=\linewidth]{p28rgliderodpp14.eps}{p28rgliderodpp14.pdf}
    \caption{P28R-200B Glide Rate of Descent, Minimum Propeller Pitch}
    \label{fig:p28rgliderodpp14}
  \end{center}
\end{figure}

\begin{figure}[!htbp]
  \begin{center}
    \jnxfig[angle=0,width=\linewidth]{p28rglidedistpp14.eps}{p28rglidedistpp14.pdf}
    \caption{P28R-200B Glide Distance, Minimum Propeller Pitch}
    \label{fig:p28rglidedistpp14}
  \end{center}
\end{figure}

\begin{figure}[!htbp]
  \begin{center}
    \jnxfig[angle=0,width=\linewidth]{p28rgliderpmpp14.eps}{p28rgliderpmpp14.pdf}
    \caption{P28R-200B Glide Windmilling RPM, Minimum Propeller Pitch}
    \label{fig:p28rgliderpmpp14}
  \end{center}
\end{figure}

\begin{figure}[!htbp]
  \begin{center}
    \jnxfig[angle=0,width=\linewidth]{p28rbestglidepp29.eps}{p28rbestglidepp29.pdf}
    \caption{P28R-200B Best Glide, Maximum Propeller Pitch}
    \label{fig:p28rbestglidepp29}
  \end{center}
\end{figure}

\begin{figure}[!htbp]
  \begin{center}
    \jnxfig[angle=0,width=\linewidth]{p28rbestglideproppp29.eps}{p28rbestglideproppp29.pdf}
    \caption{P28R-200B Best Glide Propeller Parameters, Maximum Propeller Pitch}
    \label{fig:p28rbestglideproppp29}
  \end{center}
\end{figure}

\begin{figure}[!htbp]
  \begin{center}
    \jnxfig[angle=0,width=\linewidth]{p28rgliderodpp29.eps}{p28rgliderodpp29.pdf}
    \caption{P28R-200B Glide Rate of Descent, Maximum Propeller Pitch}
    \label{fig:p28rgliderodpp29}
  \end{center}
\end{figure}

\begin{figure}[!htbp]
  \begin{center}
    \jnxfig[angle=0,width=\linewidth]{p28rglidedistpp29.eps}{p28rglidedistpp29.pdf}
    \caption{P28R-200B Glide Distance, Maximum Propeller Pitch}
    \label{fig:p28rglidedistpp29}
  \end{center}
\end{figure}

\begin{figure}[!htbp]
  \begin{center}
    \jnxfig[angle=0,width=\linewidth]{p28rgliderpmpp29.eps}{p28rgliderpmpp29.pdf}
    \caption{P28R-200B Glide Windmilling RPM, Maximum Propeller Pitch}
    \label{fig:p28rgliderpmpp29}
  \end{center}
\end{figure}
\clearpage


\subsubsection{Piper Cherokee Arrow Data}

\noindent\begin{tabular}{llr}
  NACA 652--415 Profile & & \\
  Angle where Lift Coefficient is zero & $\alpha$ & $-3^\circ$ \\
  Drag Coefficient at $\alpha$ & $c_d$ & 0.006 \\
  Drag Coefficient at $\alpha+8^\circ$ & $c_d$ & 0.008 \\
  Lift coefficient scales with $2\pi\alpha$ ($\alpha$ in rad) & & \\
  Maximum Lift Coefficient & $c_{l,max}$ & 1.2 \\
  at angle of attack & $\alpha_{max}$ & 12 \\
  Wing Area & & $160\mathrm{ft}^2$ \\
  & & $14.864\mathrm{m}^2$ \\
  MTOM & & $2600\mathrm{lb}$ \\
  & & $1179.3402\mathrm{kg}$ \\
  Rated Horsepower & & $200\mathrm{HP}$ \\
  & & $149.14\mathrm{kW}$ \\
  Rated RPM & & $2700\mathrm{min}^{-1}$ \\
\end{tabular}

\fi

\subsection{Linear Interpolation}

\noindent\begin{figure}[!htbp]
  \begin{center}
    \jnxfig[scale=0.7]{lininterp.eps}{lininterp.pdf}
    \caption{Linear Interpolation}
    \label{fig:lininterp}
  \end{center}
\end{figure}

Linear Interpolation is used extensively for weather data, to
interpolate between two points in time and two pressure levels, or
between grid points.

$0\leq\alpha\leq{}1$ and $0\leq\beta\leq{}1$.

\begin{eqnarray}
  x_0 = (1-\alpha)x_{00} + \alpha x_{01} \\
  x_1 = (1-\alpha)x_{10} + \alpha x_{11}
\end{eqnarray}

\begin{equation}
  \begin{split}
    x &= (1-\beta)x_0 + \beta x_1 = (1-\alpha)(1-\beta)x_{00} + \alpha(1-\beta) x_{01} + (1-\alpha)\beta x_{10} + \alpha\beta x_{11} \\
    &= (1-\alpha-\beta+\alpha\beta)x_{00} + (\alpha-\alpha\beta) x_{01} + (\beta-\alpha\beta) x_{10} + \alpha\beta x_{11} \\
    &= x_{00} + \alpha(x_{01}-x_{00}) + \beta(x_{10}-x_{00}) + \alpha\beta(x_{11}+x_{00}-x_{01}-x_{10}) \\
    &= p_0 + \alpha p_1 + \beta p_2 + \alpha\beta p_3
  \end{split}
\end{equation}

\begin{equation}
  \mathbf{\hat{x}} = \begin{pmatrix}
    \hat{x}_0 \\
    \hat{x}_1 \\
    \hat{x}_2 \\
    \hat{x}_3 \\
  \end{pmatrix}
\end{equation}

\begin{equation}
  \mathbf{p} = \begin{pmatrix}
    p_0 \\
    p_1 \\
    p_2 \\
    p_3 \\
  \end{pmatrix}
\end{equation}

\begin{equation}
  \mathbf{A} = \begin{pmatrix}
    1 & \alpha_0 & \beta_0 & \alpha_0\beta_0 \\
    1 & \alpha_1 & \beta_1 & \alpha_1\beta_1 \\
    1 & \alpha_2 & \beta_2 & \alpha_2\beta_2 \\
    1 & \alpha_3 & \beta_3 & \alpha_3\beta_3 \\
  \end{pmatrix}
\end{equation}

\begin{equation}
  \begin{split}
    \frac\partial{\partial\mathbf{p}} (\mathbf{\hat{x}}-\mathbf{A}\mathbf{p})^T(\mathbf{\hat{x}}-\mathbf{A}\mathbf{p})
    &= \frac\partial{\partial\mathbf{p}} (\mathbf{\hat{x}}^T\mathbf{\hat{x}}-\mathbf{p}^T\mathbf{A}^T\mathbf{\hat{x}}-\mathbf{\hat{x}}^T\mathbf{A}\mathbf{p}+\mathbf{p}^T\mathbf{A}^T\mathbf{A}\mathbf{p}) \\
    &= -\mathbf{\hat{x}}^T\mathbf{A}+\mathbf{p}^T\mathbf{A}^T\mathbf{A} = 0
  \end{split}
\end{equation}

\begin{equation}
  \mathbf{p} = (\mathbf{A}^T\mathbf{A})^{-1}\mathbf{A}^T\mathbf{\hat{x}}
\end{equation}



%\nocite{attitude:lowcostahrs}

\end{document}
